\section{Einleitung}
\lbrack Zitat (optional)\rbrack :
\begin{quote}
\glqq Was ist die Absicht eines wissenschaftlichen Buches? Es stellt Gedanken dar und will den Leser von ihrer Gültigkeit überzeugen. Darüber hinaus will der Leser auch wissen: woher kommen diese Gedanken und wohin führen sie? Mit welchen Richtungen auf anderen Gebieten hängen sie zusammen?\grqq
\pcite{}{}{benlian2013}
\end{quote}

Einer der wichtigsten Abschnitte der Arbeit ist die Einleitung, in der der Leser in das Thema eingeführt wird. Auch ein fachfremder Leser muss nach dem Lesen der Einleitung verstanden haben, warum das vorliegende Thema wichtig und erforschenswert ist. Die Einleitung sollte zum Weiterlesen animieren und das Interesse des Lesers wecken. Neben dieser Motivation der Arbeit sind die Zielsetzung und die Forschungsfragen der Arbeit zu konkretisieren. Diese werden im Laufe der Arbeit beantwortet. Abschließend folgt ein kurzer Überblick über die Arbeit.\par\medskip
In den folgenden Kapiteln wird ein typischer Aufbau einer wissenschaftlichen Arbeit dargestellt und jeweils anhand ihrer typischen Inhalte beschrieben. Dieser Aufbau ist jedoch nicht verbindlich und kann je nach Forschungsmethode stark variieren. Bitte klären Sie dies mit Ihrem jeweiligen Betreuer ab.\par\medskip
Nach der Einleitung (Kapitel 1) folgen die Grundlagen (Kapitel 2) und die Entwicklung eines Forschungsmodells (Kapitel 3). In Kapitel 4 wird die verwendete Forschungsmethode dargestellt und in Kapitel 5 die Forschungsergebnisse. Eine Diskussion der Ergebnisse findet in Kapitel 6 statt. Die Arbeit schließt mit einer abschließenden Zusammenfassung sowie mit einem Fazit und einem Ausblick (Kapitel 7).

